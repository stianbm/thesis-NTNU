% \subsection{Gameflow}
    % \label{sec:gameflow}
    % \todo{Write? It's unlikely to be used in the application as much is covered by fun keys, which is simpler}

%(summary)
%In order to compare the different solution, the two frameworks discussed earlier will be used. The virtual field trips will be placed on the \emph{SENSATIUM}- and \emph{Taxonomy} scale from Klippel et al at Penn State, discussed in \cref{sec:framework}. They will also be given a short description of their educational material, knowledge/attitude, game interactions and game controls. These descriptions are about sub-factors the \emph{LEAGUE} framework presented in \cref{sec:league}. Describing the sub-factors is not one of the metrics presented in the framework, as the sub factors are not to be evaluated at this time, only presented. These four sub-factors were chosen because they are believed to be the most distinctive features of the solutions. The four sub-factors, which can be found in LEAGUE hierarchy in \cref{fig:league}, are:
% \begin{quote}
    %     \texttt{Learning/pedagogy > L3.Learning content > 1.Syllabus matching/educational material}
        
    %     \texttt{Learning/pedagogy > L4.Learning outcome > 1.Knowledge/skills/attitude}
        
    %     \texttt{Game Factors > G3.Game Mechanics > 1.Game Interactions}
        
    %     \texttt{Game Factors > G3.Game Mechanics > 2. Game Controls}.
    % \end{quote}

%\subsection{Non-VR Applications}
    %Although more interactive ways to prepare for field courses are appearing, the most common way is to use regular documentation and maps. This is also the case for the course \emph{GEOG2012 Field Course in Physical Geography} where the students focus on building up their theoretical knowledge of the area and geographical phenomenons via articles, books and maps. The students also train on how to set up GPS base stations and use the LiDAR scanning technology beforehand, but it is done outside of the Dragvoll campus of NTNU.
        
    %\todo{This might not be relevant enough?}